\begin{frame}
	\frametitle{Symbolic Memory Addresses}
	\begin{itemize}
		\item The \texttt{LOAD} and \texttt{STORE} rules evaluate the expression representing the memory address to a value
		\begin{itemize}
			\item  that value must be a \textbf{non-negative integer} that references a particular memory cell
		\end{itemize}
		\item What are we supposed to do if the address referenced operation is an expression derived from user input?
		\begin{itemize}
			\item \textbf{Symbolic Memory Address problem}
		\end{itemize}
		\item \textbf{Sound} strategy: consider the instruction for any possible satisfying assignment 
		\item<2-> \only<2->{
			There's even worse: \textbf{aliasing} \\
			\texttt{\textbf{store} (addr1, v) \\
			z = \textbf{load} (addr2)}
		}
	\end{itemize}
\end{frame}