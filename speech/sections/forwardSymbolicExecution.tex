\section{Forward Symbolic Execution} 
One of the advantages of forward symbolic execution is that it can be used to reason about more than one input at once. For instance, in this program, only one the value 19 will cause the program to take the true branch. Forward symbolic execution can reason about the program by considering two different input classes — inputs that take the true branch, and those that take the false branch.

\begin{lstlisting}
x := 2 * get_input(src)
if x - 5 == 14 then goto 3 else goto 4
// catastrophic failure
// normal behaviour
\end{lstlisting}

The primary difference between forward symbolic execution and regular execution is that when \texttt{get\_input(·)} is evaluated symbolically, it returns a symbol instead of a concrete value. When a new symbol is first returned, there are no constraints on its value; it represents any possible value. As a result, expressions involving symbols cannot be fully evaluated to a concrete value.

Branches constrain the values of symbolic variables to the set of values that would execute the path. For example, every \texttt{if then else} statement creates two partitions of the input domain.

\subsection{Challenges and Opportunities}
Unfortunately, by examining our formal definition of this intuition, we can find several instances where our analysis breaks down.

\subsubsection{Path Selection} 
When forward symbolic execution encounters a branch, it must decide which branch to follow first. We call this the \textbf{path selection problem}.

We can think of a forward symbolic execution of an entire program as a tree in which every node represents a particular instance of the abstract machine. The analysis begins with only a root node in the tree. However, every time the analysis must fork, such as when a conditional jump is encountered, it adds as children all possible forked states to the current node. We can further explore any leaf node in the tree that has not terminated. 

Thus, forward symbolic execution needs a strategy for choosing which state to explore next. This choice is important, because loops with symbolic conditions may never terminate. 

Furthermore, a straightforward implementation of forward symbolic execution will lead to:
\begin{itemize}
	\item a running time exponential in the number of program branches, because a new interpreter is forked off at each branch point;
	\item an exponential number of formulas, which directly follows from the previous, as there is a separate formula at each branch point.
\end{itemize}
In practice, we can mitigate these problems in a number of ways:
\begin{itemize}
	\item \textbf{Path selection heuristics}:
	\begin{itemize}
		\item \textbf{Depth-First Search}: DFS employs the standard depth-first search algorithm on the state tree. The primary disadvantage of DFS is that it can get stuck in nonterminating loops with symbolic conditions if no maximum depth is specified. If this happens, then no other branches will be explored and code coverage will suffer;
		\item \textbf{Random Paths}: this kinds of strategies traverse the state tree from the root until they reach a leaf node. The random path strategy gives a higher weight to shallow states;
		\item \textbf{Concolic Testing}: concolic testing uses concrete execution to produce a trace of a program execution. Forward symbolic execution then follows the same path as the concrete execution.
	\end{itemize}
	\item \textbf{Use more and faster hardware}. Exploring multiple paths and solving formulas for each path is inherently parallelizable;
	\item \textbf{Identify independent subformulas}. By identifying logically independent subformulas and analyze them using SMT solvers;
	\item \textbf{Identify redundancies} between formulas and make them more compact.
\end{itemize}